%################################  OPCIONES DEL DOCUMENTO  ############################################

\theoremstyle{definition}
\newtheorem{teo}{Teorema}[section]
\newtheorem{coro}{Corolario}
\newtheorem{prop}{Proposición}
\newtheorem{defi}{Definición}[section]


\newtheoremstyle{ejjstyle}% Name of the style to be used
  {1em}% Space above
  {1em}% Space below
  {\itshape}% Body font
  {}% Indent amount
  {\bfseries}% Theorem head font
  {.}% Punctuation after theorem head
  {.5em}% Space after theorem head
  {\thmname{#1}\thmnumber{ #2}\thmnote{ #3}}% Theorem head specification (can be customized)

% Apply the custom theorem style to each theorem-like environment
\theoremstyle{ejjstyle}
\newtheorem{ej}{Parte}
\newtheorem{ejj}{Ejercicio}

\newcommand{\espEj}{\vspace*{0.1cm}}        %espaciado entre ejercicios
\newcommand{\espPreEj}{\vspace*{0.3cm}}     %espaciado entre titulo de seccion y ejercicios
\newcommand{\espPostEj}{\vspace*{0.4cm}}    %espaciado entre final de ejercicios y proxima seccion

%  margenes
\setmargins
{2.5cm}                    % margen izquierdo
{1cm}                    % margen superior
{16.5cm}                 % anchura del texto
{23.42cm}                % altura del texto
{15pt}                   % altura de los encabezados
{1cm}                    % espacio entre el texto y los encabezados
{0pt}                    % altura del pie de página
{2cm}                    % espacio entre el texto y el pie de página

%  encabezado y pie de pagina
\pagestyle{fancy}
\fancyhf{}
\rhead{ITBA}
\lhead{Teoría de Circuitos II}
\chead{Caviglia}
\rfoot{\thepage}


\usepackage{hyperref}
\hypersetup{
    colorlinks=true,
    linkcolor=black,
    filecolor=magenta,      
    urlcolor=blue,
    pdfpagemode=FullScreen,
}

%   Settings de pgfplots
\pgfplotsset{
    standard/.style={
    axis line style = thick,
    trig format=rad,
    enlargelimits,
    axis x line=middle,
    axis y line=middle,
    enlarge x limits=0.15,
    enlarge y limits=0.15,
    every axis x label/.style={at={(current axis.right of origin)},anchor=north west},
    every axis y label/.style={at={(current axis.above origin)},anchor=south east}
    }
}